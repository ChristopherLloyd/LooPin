\documentclass{article}%
\usepackage[T1]{fontenc}%
\usepackage[utf8]{inputenc}%
\usepackage{lmodern}%
\usepackage{textcomp}%
\usepackage{lastpage}%
\usepackage{geometry}%
\usepackage{tabularx}%
\usepackage{booktabs}%
\usepackage[dvipsnames]{xcolor}
\usepackage{tikz}
\usepackage{tkz-graph}
\usepackage{tkz-berge}
\usetikzlibrary{arrows,shapes}
\usepackage[matrix,arrow,curve,cmtip]{xy}
\usepackage{svg}
\usepackage{multicol}
\usepackage{float}
\usepackage{graphicx}
\usepackage[shortlabels]{enumitem}
\usepackage{hyperref}
\usepackage[nottoc,numbib]{tocbibind}
\geometry{tmargin=2cm,lmargin=2cm,rmargin=2cm,bmargin=2cm}%
%
%
%
\definecolor{red0}{rgb}{1.0,0,0}
\definecolor{green0}{rgb}{0,1.0,0}
%
%
%
\pdfsuppresswarningpagegroup=1

\setcounter{tocdepth}{2}

\title{Pinning poset changes drastically under Reidemeister III}

\author{Christopher-Lloyd Simon and Ben Stucky}

\begin{document}%

\section{Pinning poset changes drastically under Reidemeister III}

\subsubsection{[(1, 7, 2, 6), (24, 7, 1, 8), (5, 10, 6, 11), (4, 12, 5, 11), (9, 15, 10, 14), (8, 15, 9, 16), (13, 18, 14, 19), (12, 20, 13, 19), (2, 22, 3, 21), (3, 20, 4, 21), (17, 23, 18, 22), (16, 23, 17, 24)]}

{\small\noindent PD code drawn by \texttt{SnapPy}: [(3, 22, 4, 23), (21, 4, 22, 5), (2, 7, 3, 8), (8, 1, 9, 2), (11, 6, 12, 7), (5, 12, 6, 13), (10, 15, 11, 16), (16, 9, 17, 10), (24, 17, 1, 18), (18, 23, 19, 24), (19, 14, 20, 15), (13, 20, 14, 21)]}

{\small\noindent Planar representation generated by \texttt{plantri}: -}

\begin{multicols}{2}
{\normalsize \noindent\textbf{Total optimal pinning sets:} 1

\noindent\textbf{Total minimal pinning sets:} 1

\noindent\textbf{Total pinning sets:} 256

\noindent\textbf{Pinning number:} 6

}
\columnbreak

{\normalsize \noindent\textbf{Average optimal degree:} 2.0

\noindent\textbf{Average minimal degree:} 2.0

\noindent\textbf{Average overall degree:} 2.97

}
\end{multicols}

\begin{table}[ht]
	\caption{Pinning sets/average degree by cardinal}
	\centering
	\renewcommand{\arraystretch}{1.5}
	\begin{tabularx}{\textwidth}{lXXXXXXXXXXX}
		\toprule
			Cardinal & 6 & 7 & 8 & 9 & 10 & 11 & 12 & 13 & 14 & Total\\
			\hline
			Optimal pinning sets & 1 & 0 & 0 & 0 & 0 & 0 & 0 & 0 & 0 & 1 \\
			Minimal (suboptimal) pinning sets & 0 & 0 & 0 & 0 & 0 & 0 & 0 & 0 & 0 & 0 \\
			Nonminimal pinning sets & 0 & 8 & 28 & 56 & 70 & 56 & 28 & 8 & 1 & 255 \\
			Average degree & 2.0 & 2.36 & 2.62 & 2.83 & 3.0 & 3.14 & 3.25 & 3.35 & 3.43 &  \\
		\bottomrule \\ 
	\end{tabularx}
\end{table}

\begin{multicols}{2}
\begin{figure}[H]
\centering
\includesvg[inkscapelatex=false,width=250pt]{tex/img/[(1, 7, 2, 6), (24, 7, 1, 8), (5, 10, 6, 11), (4, 12, 5, 11), (9, 15, 10, 14), (8, 15, 9, 16), (13, 18, 14, 19), (12, 20, 13, 19), (2, 22, 3, 21), (3, 20, 4, 21), (17, 23, 18, 22), (16, 23, 17, 24)].svg}
\caption{\texttt{SnapPy} multiloop plot.}
\label{fig:tex/img/[(1, 7, 2, 6), (24, 7, 1, 8), (5, 10, 6, 11), (4, 12, 5, 11), (9, 15, 10, 14), (8, 15, 9, 16), (13, 18, 14, 19), (12, 20, 13, 19), (2, 22, 3, 21), (3, 20, 4, 21), (17, 23, 18, 22), (16, 23, 17, 24)].svg}
\end{figure}
\columnbreak

\begin{figure}[H]
\centering
\scalebox{0.8}{%% Creator: Matplotlib, PGF backend
%%
%% To include the figure in your LaTeX document, write
%%   \input{<filename>.pgf}
%%
%% Make sure the required packages are loaded in your preamble
%%   \usepackage{pgf}
%%
%% Also ensure that all the required font packages are loaded; for instance,
%% the lmodern package is sometimes necessary when using math font.
%%   \usepackage{lmodern}
%%
%% Figures using additional raster images can only be included by \input if
%% they are in the same directory as the main LaTeX file. For loading figures
%% from other directories you can use the `import` package
%%   \usepackage{import}
%%
%% and then include the figures with
%%   \import{<path to file>}{<filename>.pgf}
%%
%% Matplotlib used the following preamble
%%   
%%   \usepackage{fontspec}
%%   \setmainfont{DejaVuSerif.ttf}[Path=\detokenize{/home/ben/sage/local/var/lib/sage/venv-python3.10/lib/python3.10/site-packages/matplotlib/mpl-data/fonts/ttf/}]
%%   \setsansfont{DejaVuSans.ttf}[Path=\detokenize{/home/ben/sage/local/var/lib/sage/venv-python3.10/lib/python3.10/site-packages/matplotlib/mpl-data/fonts/ttf/}]
%%   \setmonofont{DejaVuSansMono.ttf}[Path=\detokenize{/home/ben/sage/local/var/lib/sage/venv-python3.10/lib/python3.10/site-packages/matplotlib/mpl-data/fonts/ttf/}]
%%   \makeatletter\@ifpackageloaded{underscore}{}{\usepackage[strings]{underscore}}\makeatother
%%
\begingroup%
\makeatletter%
\begin{pgfpicture}%
\pgfpathrectangle{\pgfpointorigin}{\pgfqpoint{0.376731in}{4.700000in}}%
\pgfusepath{use as bounding box, clip}%
\begin{pgfscope}%
\pgfsetbuttcap%
\pgfsetmiterjoin%
\definecolor{currentfill}{rgb}{1.000000,1.000000,1.000000}%
\pgfsetfillcolor{currentfill}%
\pgfsetlinewidth{0.000000pt}%
\definecolor{currentstroke}{rgb}{1.000000,1.000000,1.000000}%
\pgfsetstrokecolor{currentstroke}%
\pgfsetdash{}{0pt}%
\pgfpathmoveto{\pgfqpoint{0.000000in}{0.000000in}}%
\pgfpathlineto{\pgfqpoint{0.376731in}{0.000000in}}%
\pgfpathlineto{\pgfqpoint{0.376731in}{4.700000in}}%
\pgfpathlineto{\pgfqpoint{0.000000in}{4.700000in}}%
\pgfpathlineto{\pgfqpoint{0.000000in}{0.000000in}}%
\pgfpathclose%
\pgfusepath{fill}%
\end{pgfscope}%
\begin{pgfscope}%
\pgfsetbuttcap%
\pgfsetmiterjoin%
\definecolor{currentfill}{rgb}{1.000000,1.000000,1.000000}%
\pgfsetfillcolor{currentfill}%
\pgfsetlinewidth{0.000000pt}%
\definecolor{currentstroke}{rgb}{0.000000,0.000000,0.000000}%
\pgfsetstrokecolor{currentstroke}%
\pgfsetstrokeopacity{0.000000}%
\pgfsetdash{}{0pt}%
\pgfpathmoveto{\pgfqpoint{0.188321in}{0.100000in}}%
\pgfpathlineto{\pgfqpoint{0.190645in}{0.100000in}}%
\pgfpathlineto{\pgfqpoint{0.190645in}{4.600000in}}%
\pgfpathlineto{\pgfqpoint{0.188321in}{4.600000in}}%
\pgfpathlineto{\pgfqpoint{0.188321in}{0.100000in}}%
\pgfpathclose%
\pgfusepath{fill}%
\end{pgfscope}%
\begin{pgfscope}%
\pgfpathrectangle{\pgfqpoint{0.188321in}{0.100000in}}{\pgfqpoint{0.002324in}{4.500000in}}%
\pgfusepath{clip}%
\pgfsetrectcap%
\pgfsetroundjoin%
\pgfsetlinewidth{2.007500pt}%
\definecolor{currentstroke}{rgb}{0.875000,0.875000,0.875000}%
\pgfsetstrokecolor{currentstroke}%
\pgfsetdash{}{0pt}%
\pgfpathmoveto{\pgfqpoint{0.190600in}{0.186538in}}%
\pgfpathlineto{\pgfqpoint{0.188365in}{4.513462in}}%
\pgfusepath{stroke}%
\end{pgfscope}%
\begin{pgfscope}%
\pgfsetbuttcap%
\pgfsetroundjoin%
\definecolor{currentfill}{rgb}{0.700000,0.700000,1.000000}%
\pgfsetfillcolor{currentfill}%
\pgfsetlinewidth{1.003750pt}%
\definecolor{currentstroke}{rgb}{0.000000,0.000000,0.000000}%
\pgfsetstrokecolor{currentstroke}%
\pgfsetdash{}{0pt}%
\pgfpathmoveto{\pgfqpoint{0.188365in}{4.415252in}}%
\pgfpathcurveto{\pgfqpoint{0.214411in}{4.415252in}}{\pgfqpoint{0.239393in}{4.425600in}}{\pgfqpoint{0.257810in}{4.444017in}}%
\pgfpathcurveto{\pgfqpoint{0.276227in}{4.462434in}}{\pgfqpoint{0.286575in}{4.487416in}}{\pgfqpoint{0.286575in}{4.513462in}}%
\pgfpathcurveto{\pgfqpoint{0.286575in}{4.539507in}}{\pgfqpoint{0.276227in}{4.564489in}}{\pgfqpoint{0.257810in}{4.582906in}}%
\pgfpathcurveto{\pgfqpoint{0.239393in}{4.601323in}}{\pgfqpoint{0.214411in}{4.611671in}}{\pgfqpoint{0.188365in}{4.611671in}}%
\pgfpathcurveto{\pgfqpoint{0.162320in}{4.611671in}}{\pgfqpoint{0.137338in}{4.601323in}}{\pgfqpoint{0.118921in}{4.582906in}}%
\pgfpathcurveto{\pgfqpoint{0.100504in}{4.564489in}}{\pgfqpoint{0.090156in}{4.539507in}}{\pgfqpoint{0.090156in}{4.513462in}}%
\pgfpathcurveto{\pgfqpoint{0.090156in}{4.487416in}}{\pgfqpoint{0.100504in}{4.462434in}}{\pgfqpoint{0.118921in}{4.444017in}}%
\pgfpathcurveto{\pgfqpoint{0.137338in}{4.425600in}}{\pgfqpoint{0.162320in}{4.415252in}}{\pgfqpoint{0.188365in}{4.415252in}}%
\pgfpathlineto{\pgfqpoint{0.188365in}{4.415252in}}%
\pgfpathclose%
\pgfusepath{stroke,fill}%
\end{pgfscope}%
\begin{pgfscope}%
\pgfsetbuttcap%
\pgfsetroundjoin%
\definecolor{currentfill}{rgb}{1.000000,0.000000,0.000000}%
\pgfsetfillcolor{currentfill}%
\pgfsetlinewidth{1.003750pt}%
\definecolor{currentstroke}{rgb}{0.000000,0.000000,0.000000}%
\pgfsetstrokecolor{currentstroke}%
\pgfsetdash{}{0pt}%
\pgfpathmoveto{\pgfqpoint{0.190600in}{0.088329in}}%
\pgfpathcurveto{\pgfqpoint{0.216646in}{0.088329in}}{\pgfqpoint{0.241628in}{0.098677in}}{\pgfqpoint{0.260045in}{0.117094in}}%
\pgfpathcurveto{\pgfqpoint{0.278462in}{0.135511in}}{\pgfqpoint{0.288810in}{0.160493in}}{\pgfqpoint{0.288810in}{0.186538in}}%
\pgfpathcurveto{\pgfqpoint{0.288810in}{0.212584in}}{\pgfqpoint{0.278462in}{0.237566in}}{\pgfqpoint{0.260045in}{0.255983in}}%
\pgfpathcurveto{\pgfqpoint{0.241628in}{0.274400in}}{\pgfqpoint{0.216646in}{0.284748in}}{\pgfqpoint{0.190600in}{0.284748in}}%
\pgfpathcurveto{\pgfqpoint{0.164555in}{0.284748in}}{\pgfqpoint{0.139573in}{0.274400in}}{\pgfqpoint{0.121156in}{0.255983in}}%
\pgfpathcurveto{\pgfqpoint{0.102739in}{0.237566in}}{\pgfqpoint{0.092391in}{0.212584in}}{\pgfqpoint{0.092391in}{0.186538in}}%
\pgfpathcurveto{\pgfqpoint{0.092391in}{0.160493in}}{\pgfqpoint{0.102739in}{0.135511in}}{\pgfqpoint{0.121156in}{0.117094in}}%
\pgfpathcurveto{\pgfqpoint{0.139573in}{0.098677in}}{\pgfqpoint{0.164555in}{0.088329in}}{\pgfqpoint{0.190600in}{0.088329in}}%
\pgfpathlineto{\pgfqpoint{0.190600in}{0.088329in}}%
\pgfpathclose%
\pgfusepath{stroke,fill}%
\end{pgfscope}%
\begin{pgfscope}%
\definecolor{textcolor}{rgb}{0.000000,0.000000,0.000000}%
\pgfsetstrokecolor{textcolor}%
\pgfsetfillcolor{textcolor}%
\pgftext[x=0.190600in,y=0.186538in,,]{\color{textcolor}\sffamily\fontsize{10.000000}{12.000000}\selectfont 6}%
\end{pgfscope}%
\begin{pgfscope}%
\definecolor{textcolor}{rgb}{0.000000,0.000000,0.000000}%
\pgfsetstrokecolor{textcolor}%
\pgfsetfillcolor{textcolor}%
\pgftext[x=0.188365in,y=4.513462in,,]{\color{textcolor}\sffamily\fontsize{10.000000}{12.000000}\selectfont 14}%
\end{pgfscope}%
\end{pgfpicture}%
\makeatother%
\endgroup%
}
\caption{Minimal join sub-semi-lattice of minimal pinning sets.}
\label{fig:tex/img/[(1, 7, 2, 6), (24, 7, 1, 8), (5, 10, 6, 11), (4, 12, 5, 11), (9, 15, 10, 14), (8, 15, 9, 16), (13, 18, 14, 19), (12, 20, 13, 19), (2, 22, 3, 21), (3, 20, 4, 21), (17, 23, 18, 22), (16, 23, 17, 24)].pgf}
\end{figure}
\end{multicols}

\newpage

\subsubsection{[(1, 7, 2, 6), (24, 7, 1, 8), (5, 10, 6, 11), (3, 13, 4, 12), (9, 15, 10, 14), (8, 15, 9, 16), (13, 18, 14, 19), (4, 19, 5, 20), (2, 22, 3, 21), (11, 21, 12, 20), (17, 23, 18, 22), (16, 23, 17, 24)]}

{\small\noindent PD code drawn by \texttt{SnapPy}: [(4, 23, 5, 24), (22, 5, 23, 6), (3, 8, 4, 9), (10, 1, 11, 2), (12, 7, 13, 8), (6, 13, 7, 14), (11, 16, 12, 17), (2, 17, 3, 18), (18, 9, 19, 10), (19, 24, 20, 1), (20, 15, 21, 16), (14, 21, 15, 22)]}

{\small\noindent Planar representation generated by \texttt{plantri}: -}

\begin{multicols}{2}
{\normalsize \noindent\textbf{Total optimal pinning sets:} 1

\noindent\textbf{Total minimal pinning sets:} 2

\noindent\textbf{Total pinning sets:} 1152

\noindent\textbf{Pinning number:} 4

}
\columnbreak

{\normalsize \noindent\textbf{Average optimal degree:} 2.25

\noindent\textbf{Average minimal degree:} 2.38

\noindent\textbf{Average overall degree:} 3.14

}
\end{multicols}

\begin{table}[ht]
	\caption{Pinning sets/average degree by cardinal}
	\centering
	\renewcommand{\arraystretch}{1.5}
	\begin{tabularx}{\textwidth}{lXXXXXXXXXXXXX}
		\toprule
			Cardinal & 4 & 5 & 6 & 7 & 8 & 9 & 10 & 11 & 12 & 13 & 14 & Total\\
			\hline
			Optimal pinning sets & 1 & 0 & 0 & 0 & 0 & 0 & 0 & 0 & 0 & 0 & 0 & 1 \\
			Minimal (suboptimal) pinning sets & 0 & 0 & 1 & 0 & 0 & 0 & 0 & 0 & 0 & 0 & 0 & 1 \\
			Nonminimal pinning sets & 0 & 10 & 45 & 127 & 231 & 287 & 245 & 141 & 52 & 11 & 1 & 1150 \\
			Average degree & 2.25 & 2.58 & 2.79 & 2.95 & 3.06 & 3.16 & 3.24 & 3.3 & 3.36 & 3.4 & 3.43 &  \\
		\bottomrule \\ 
	\end{tabularx}
\end{table}

\begin{multicols}{2}
\begin{figure}[H]
\centering
\includesvg[inkscapelatex=false,width=250pt]{tex/img/[(1, 7, 2, 6), (24, 7, 1, 8), (5, 10, 6, 11), (3, 13, 4, 12), (9, 15, 10, 14), (8, 15, 9, 16), (13, 18, 14, 19), (4, 19, 5, 20), (2, 22, 3, 21), (11, 21, 12, 20), (17, 23, 18, 22), (16, 23, 17, 24)].svg}
\caption{\texttt{SnapPy} multiloop plot.}
\label{fig:tex/img/[(1, 7, 2, 6), (24, 7, 1, 8), (5, 10, 6, 11), (3, 13, 4, 12), (9, 15, 10, 14), (8, 15, 9, 16), (13, 18, 14, 19), (4, 19, 5, 20), (2, 22, 3, 21), (11, 21, 12, 20), (17, 23, 18, 22), (16, 23, 17, 24)].svg}
\end{figure}
\columnbreak

\begin{figure}[H]
\centering
\scalebox{0.8}{%% Creator: Matplotlib, PGF backend
%%
%% To include the figure in your LaTeX document, write
%%   \input{<filename>.pgf}
%%
%% Make sure the required packages are loaded in your preamble
%%   \usepackage{pgf}
%%
%% Also ensure that all the required font packages are loaded; for instance,
%% the lmodern package is sometimes necessary when using math font.
%%   \usepackage{lmodern}
%%
%% Figures using additional raster images can only be included by \input if
%% they are in the same directory as the main LaTeX file. For loading figures
%% from other directories you can use the `import` package
%%   \usepackage{import}
%%
%% and then include the figures with
%%   \import{<path to file>}{<filename>.pgf}
%%
%% Matplotlib used the following preamble
%%   
%%   \usepackage{fontspec}
%%   \setmainfont{DejaVuSerif.ttf}[Path=\detokenize{/home/ben/sage/local/var/lib/sage/venv-python3.10/lib/python3.10/site-packages/matplotlib/mpl-data/fonts/ttf/}]
%%   \setsansfont{DejaVuSans.ttf}[Path=\detokenize{/home/ben/sage/local/var/lib/sage/venv-python3.10/lib/python3.10/site-packages/matplotlib/mpl-data/fonts/ttf/}]
%%   \setmonofont{DejaVuSansMono.ttf}[Path=\detokenize{/home/ben/sage/local/var/lib/sage/venv-python3.10/lib/python3.10/site-packages/matplotlib/mpl-data/fonts/ttf/}]
%%   \makeatletter\@ifpackageloaded{underscore}{}{\usepackage[strings]{underscore}}\makeatother
%%
\begingroup%
\makeatletter%
\begin{pgfpicture}%
\pgfpathrectangle{\pgfpointorigin}{\pgfqpoint{0.391769in}{4.700000in}}%
\pgfusepath{use as bounding box, clip}%
\begin{pgfscope}%
\pgfsetbuttcap%
\pgfsetmiterjoin%
\definecolor{currentfill}{rgb}{1.000000,1.000000,1.000000}%
\pgfsetfillcolor{currentfill}%
\pgfsetlinewidth{0.000000pt}%
\definecolor{currentstroke}{rgb}{1.000000,1.000000,1.000000}%
\pgfsetstrokecolor{currentstroke}%
\pgfsetdash{}{0pt}%
\pgfpathmoveto{\pgfqpoint{0.000000in}{0.000000in}}%
\pgfpathlineto{\pgfqpoint{0.391769in}{0.000000in}}%
\pgfpathlineto{\pgfqpoint{0.391769in}{4.700000in}}%
\pgfpathlineto{\pgfqpoint{0.000000in}{4.700000in}}%
\pgfpathlineto{\pgfqpoint{0.000000in}{0.000000in}}%
\pgfpathclose%
\pgfusepath{fill}%
\end{pgfscope}%
\begin{pgfscope}%
\pgfsetbuttcap%
\pgfsetmiterjoin%
\definecolor{currentfill}{rgb}{1.000000,1.000000,1.000000}%
\pgfsetfillcolor{currentfill}%
\pgfsetlinewidth{0.000000pt}%
\definecolor{currentstroke}{rgb}{0.000000,0.000000,0.000000}%
\pgfsetstrokecolor{currentstroke}%
\pgfsetstrokeopacity{0.000000}%
\pgfsetdash{}{0pt}%
\pgfpathmoveto{\pgfqpoint{0.187181in}{0.100000in}}%
\pgfpathlineto{\pgfqpoint{0.248771in}{0.100000in}}%
\pgfpathlineto{\pgfqpoint{0.248771in}{4.600000in}}%
\pgfpathlineto{\pgfqpoint{0.187181in}{4.600000in}}%
\pgfpathlineto{\pgfqpoint{0.187181in}{0.100000in}}%
\pgfpathclose%
\pgfusepath{fill}%
\end{pgfscope}%
\begin{pgfscope}%
\pgfpathrectangle{\pgfqpoint{0.187181in}{0.100000in}}{\pgfqpoint{0.061590in}{4.500000in}}%
\pgfusepath{clip}%
\pgfsetrectcap%
\pgfsetroundjoin%
\pgfsetlinewidth{2.007500pt}%
\definecolor{currentstroke}{rgb}{0.285714,0.285714,0.285714}%
\pgfsetstrokecolor{currentstroke}%
\pgfsetdash{}{0pt}%
\pgfpathmoveto{\pgfqpoint{0.214069in}{0.186538in}}%
\pgfpathlineto{\pgfqpoint{0.226656in}{1.484615in}}%
\pgfusepath{stroke}%
\end{pgfscope}%
\begin{pgfscope}%
\pgfpathrectangle{\pgfqpoint{0.187181in}{0.100000in}}{\pgfqpoint{0.061590in}{4.500000in}}%
\pgfusepath{clip}%
\pgfsetrectcap%
\pgfsetroundjoin%
\pgfsetlinewidth{2.007500pt}%
\definecolor{currentstroke}{rgb}{0.000000,0.000000,0.000000}%
\pgfsetstrokecolor{currentstroke}%
\pgfsetdash{}{0pt}%
\pgfpathmoveto{\pgfqpoint{0.247587in}{1.051923in}}%
\pgfpathlineto{\pgfqpoint{0.226656in}{1.484615in}}%
\pgfusepath{stroke}%
\end{pgfscope}%
\begin{pgfscope}%
\pgfpathrectangle{\pgfqpoint{0.187181in}{0.100000in}}{\pgfqpoint{0.061590in}{4.500000in}}%
\pgfusepath{clip}%
\pgfsetrectcap%
\pgfsetroundjoin%
\pgfsetlinewidth{2.007500pt}%
\definecolor{currentstroke}{rgb}{0.857143,0.857143,0.857143}%
\pgfsetstrokecolor{currentstroke}%
\pgfsetdash{}{0pt}%
\pgfpathmoveto{\pgfqpoint{0.226656in}{1.484615in}}%
\pgfpathlineto{\pgfqpoint{0.188365in}{4.513462in}}%
\pgfusepath{stroke}%
\end{pgfscope}%
\begin{pgfscope}%
\pgfsetbuttcap%
\pgfsetroundjoin%
\definecolor{currentfill}{rgb}{0.700000,0.700000,1.000000}%
\pgfsetfillcolor{currentfill}%
\pgfsetlinewidth{1.003750pt}%
\definecolor{currentstroke}{rgb}{0.000000,0.000000,0.000000}%
\pgfsetstrokecolor{currentstroke}%
\pgfsetdash{}{0pt}%
\pgfpathmoveto{\pgfqpoint{0.226656in}{1.386406in}}%
\pgfpathcurveto{\pgfqpoint{0.252702in}{1.386406in}}{\pgfqpoint{0.277684in}{1.396754in}}{\pgfqpoint{0.296101in}{1.415171in}}%
\pgfpathcurveto{\pgfqpoint{0.314518in}{1.433588in}}{\pgfqpoint{0.324866in}{1.458570in}}{\pgfqpoint{0.324866in}{1.484615in}}%
\pgfpathcurveto{\pgfqpoint{0.324866in}{1.510661in}}{\pgfqpoint{0.314518in}{1.535643in}}{\pgfqpoint{0.296101in}{1.554060in}}%
\pgfpathcurveto{\pgfqpoint{0.277684in}{1.572477in}}{\pgfqpoint{0.252702in}{1.582825in}}{\pgfqpoint{0.226656in}{1.582825in}}%
\pgfpathcurveto{\pgfqpoint{0.200611in}{1.582825in}}{\pgfqpoint{0.175629in}{1.572477in}}{\pgfqpoint{0.157212in}{1.554060in}}%
\pgfpathcurveto{\pgfqpoint{0.138795in}{1.535643in}}{\pgfqpoint{0.128447in}{1.510661in}}{\pgfqpoint{0.128447in}{1.484615in}}%
\pgfpathcurveto{\pgfqpoint{0.128447in}{1.458570in}}{\pgfqpoint{0.138795in}{1.433588in}}{\pgfqpoint{0.157212in}{1.415171in}}%
\pgfpathcurveto{\pgfqpoint{0.175629in}{1.396754in}}{\pgfqpoint{0.200611in}{1.386406in}}{\pgfqpoint{0.226656in}{1.386406in}}%
\pgfpathlineto{\pgfqpoint{0.226656in}{1.386406in}}%
\pgfpathclose%
\pgfusepath{stroke,fill}%
\end{pgfscope}%
\begin{pgfscope}%
\pgfsetbuttcap%
\pgfsetroundjoin%
\definecolor{currentfill}{rgb}{0.700000,0.700000,1.000000}%
\pgfsetfillcolor{currentfill}%
\pgfsetlinewidth{1.003750pt}%
\definecolor{currentstroke}{rgb}{0.000000,0.000000,0.000000}%
\pgfsetstrokecolor{currentstroke}%
\pgfsetdash{}{0pt}%
\pgfpathmoveto{\pgfqpoint{0.188365in}{4.415252in}}%
\pgfpathcurveto{\pgfqpoint{0.214411in}{4.415252in}}{\pgfqpoint{0.239393in}{4.425600in}}{\pgfqpoint{0.257810in}{4.444017in}}%
\pgfpathcurveto{\pgfqpoint{0.276227in}{4.462434in}}{\pgfqpoint{0.286575in}{4.487416in}}{\pgfqpoint{0.286575in}{4.513462in}}%
\pgfpathcurveto{\pgfqpoint{0.286575in}{4.539507in}}{\pgfqpoint{0.276227in}{4.564489in}}{\pgfqpoint{0.257810in}{4.582906in}}%
\pgfpathcurveto{\pgfqpoint{0.239393in}{4.601323in}}{\pgfqpoint{0.214411in}{4.611671in}}{\pgfqpoint{0.188365in}{4.611671in}}%
\pgfpathcurveto{\pgfqpoint{0.162320in}{4.611671in}}{\pgfqpoint{0.137338in}{4.601323in}}{\pgfqpoint{0.118921in}{4.582906in}}%
\pgfpathcurveto{\pgfqpoint{0.100504in}{4.564489in}}{\pgfqpoint{0.090156in}{4.539507in}}{\pgfqpoint{0.090156in}{4.513462in}}%
\pgfpathcurveto{\pgfqpoint{0.090156in}{4.487416in}}{\pgfqpoint{0.100504in}{4.462434in}}{\pgfqpoint{0.118921in}{4.444017in}}%
\pgfpathcurveto{\pgfqpoint{0.137338in}{4.425600in}}{\pgfqpoint{0.162320in}{4.415252in}}{\pgfqpoint{0.188365in}{4.415252in}}%
\pgfpathlineto{\pgfqpoint{0.188365in}{4.415252in}}%
\pgfpathclose%
\pgfusepath{stroke,fill}%
\end{pgfscope}%
\begin{pgfscope}%
\pgfsetbuttcap%
\pgfsetroundjoin%
\definecolor{currentfill}{rgb}{0.000000,1.000000,0.000000}%
\pgfsetfillcolor{currentfill}%
\pgfsetlinewidth{1.003750pt}%
\definecolor{currentstroke}{rgb}{0.000000,0.000000,0.000000}%
\pgfsetstrokecolor{currentstroke}%
\pgfsetdash{}{0pt}%
\pgfpathmoveto{\pgfqpoint{0.247587in}{0.953714in}}%
\pgfpathcurveto{\pgfqpoint{0.273632in}{0.953714in}}{\pgfqpoint{0.298614in}{0.964062in}}{\pgfqpoint{0.317031in}{0.982479in}}%
\pgfpathcurveto{\pgfqpoint{0.335448in}{1.000896in}}{\pgfqpoint{0.345796in}{1.025878in}}{\pgfqpoint{0.345796in}{1.051923in}}%
\pgfpathcurveto{\pgfqpoint{0.345796in}{1.077968in}}{\pgfqpoint{0.335448in}{1.102951in}}{\pgfqpoint{0.317031in}{1.121368in}}%
\pgfpathcurveto{\pgfqpoint{0.298614in}{1.139784in}}{\pgfqpoint{0.273632in}{1.150132in}}{\pgfqpoint{0.247587in}{1.150132in}}%
\pgfpathcurveto{\pgfqpoint{0.221541in}{1.150132in}}{\pgfqpoint{0.196559in}{1.139784in}}{\pgfqpoint{0.178142in}{1.121368in}}%
\pgfpathcurveto{\pgfqpoint{0.159725in}{1.102951in}}{\pgfqpoint{0.149377in}{1.077968in}}{\pgfqpoint{0.149377in}{1.051923in}}%
\pgfpathcurveto{\pgfqpoint{0.149377in}{1.025878in}}{\pgfqpoint{0.159725in}{1.000896in}}{\pgfqpoint{0.178142in}{0.982479in}}%
\pgfpathcurveto{\pgfqpoint{0.196559in}{0.964062in}}{\pgfqpoint{0.221541in}{0.953714in}}{\pgfqpoint{0.247587in}{0.953714in}}%
\pgfpathlineto{\pgfqpoint{0.247587in}{0.953714in}}%
\pgfpathclose%
\pgfusepath{stroke,fill}%
\end{pgfscope}%
\begin{pgfscope}%
\pgfsetbuttcap%
\pgfsetroundjoin%
\definecolor{currentfill}{rgb}{1.000000,0.000000,0.000000}%
\pgfsetfillcolor{currentfill}%
\pgfsetlinewidth{1.003750pt}%
\definecolor{currentstroke}{rgb}{0.000000,0.000000,0.000000}%
\pgfsetstrokecolor{currentstroke}%
\pgfsetdash{}{0pt}%
\pgfpathmoveto{\pgfqpoint{0.214069in}{0.088329in}}%
\pgfpathcurveto{\pgfqpoint{0.240114in}{0.088329in}}{\pgfqpoint{0.265096in}{0.098677in}}{\pgfqpoint{0.283513in}{0.117094in}}%
\pgfpathcurveto{\pgfqpoint{0.301930in}{0.135511in}}{\pgfqpoint{0.312278in}{0.160493in}}{\pgfqpoint{0.312278in}{0.186538in}}%
\pgfpathcurveto{\pgfqpoint{0.312278in}{0.212584in}}{\pgfqpoint{0.301930in}{0.237566in}}{\pgfqpoint{0.283513in}{0.255983in}}%
\pgfpathcurveto{\pgfqpoint{0.265096in}{0.274400in}}{\pgfqpoint{0.240114in}{0.284748in}}{\pgfqpoint{0.214069in}{0.284748in}}%
\pgfpathcurveto{\pgfqpoint{0.188023in}{0.284748in}}{\pgfqpoint{0.163041in}{0.274400in}}{\pgfqpoint{0.144624in}{0.255983in}}%
\pgfpathcurveto{\pgfqpoint{0.126208in}{0.237566in}}{\pgfqpoint{0.115860in}{0.212584in}}{\pgfqpoint{0.115860in}{0.186538in}}%
\pgfpathcurveto{\pgfqpoint{0.115860in}{0.160493in}}{\pgfqpoint{0.126208in}{0.135511in}}{\pgfqpoint{0.144624in}{0.117094in}}%
\pgfpathcurveto{\pgfqpoint{0.163041in}{0.098677in}}{\pgfqpoint{0.188023in}{0.088329in}}{\pgfqpoint{0.214069in}{0.088329in}}%
\pgfpathlineto{\pgfqpoint{0.214069in}{0.088329in}}%
\pgfpathclose%
\pgfusepath{stroke,fill}%
\end{pgfscope}%
\begin{pgfscope}%
\definecolor{textcolor}{rgb}{0.000000,0.000000,0.000000}%
\pgfsetstrokecolor{textcolor}%
\pgfsetfillcolor{textcolor}%
\pgftext[x=0.214069in,y=0.186538in,,]{\color{textcolor}\sffamily\fontsize{10.000000}{12.000000}\selectfont 4}%
\end{pgfscope}%
\begin{pgfscope}%
\definecolor{textcolor}{rgb}{0.000000,0.000000,0.000000}%
\pgfsetstrokecolor{textcolor}%
\pgfsetfillcolor{textcolor}%
\pgftext[x=0.247587in,y=1.051923in,,]{\color{textcolor}\sffamily\fontsize{10.000000}{12.000000}\selectfont 6}%
\end{pgfscope}%
\begin{pgfscope}%
\definecolor{textcolor}{rgb}{0.000000,0.000000,0.000000}%
\pgfsetstrokecolor{textcolor}%
\pgfsetfillcolor{textcolor}%
\pgftext[x=0.226656in,y=1.484615in,,]{\color{textcolor}\sffamily\fontsize{10.000000}{12.000000}\selectfont 7}%
\end{pgfscope}%
\begin{pgfscope}%
\definecolor{textcolor}{rgb}{0.000000,0.000000,0.000000}%
\pgfsetstrokecolor{textcolor}%
\pgfsetfillcolor{textcolor}%
\pgftext[x=0.188365in,y=4.513462in,,]{\color{textcolor}\sffamily\fontsize{10.000000}{12.000000}\selectfont 14}%
\end{pgfscope}%
\end{pgfpicture}%
\makeatother%
\endgroup%
}
\caption{Minimal join sub-semi-lattice of minimal pinning sets.}
\label{fig:tex/img/[(1, 7, 2, 6), (24, 7, 1, 8), (5, 10, 6, 11), (3, 13, 4, 12), (9, 15, 10, 14), (8, 15, 9, 16), (13, 18, 14, 19), (4, 19, 5, 20), (2, 22, 3, 21), (11, 21, 12, 20), (17, 23, 18, 22), (16, 23, 17, 24)].pgf}
\end{figure}
\end{multicols}

\newpage


\end{document}